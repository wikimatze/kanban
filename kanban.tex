%% config
\def\home{../../styles}

%% documentclass
\input{\home/documentclass_normal_oneside}

%% generell-styling
\input{\home/style_proggen}

%% meta-tags for pdf
\newcommand{\pdfauthor}{Matthias Günther}
\newcommand{\pdftitle}{Kanban}
\newcommand{\pdfsubject}{Aufzeichnungen zur Fortbildung}
\newcommand{\pdfkeywords}{ruby}
\newcommand{\motto}{Kanban ist eventuell besser als Scrum}
\newcommand{\tutor}{}
\newcommand{\disclaimer}{(Die Autoren übernehmen keine Garantie und Haftung
für die Korrektheit des Skriptes. Das Skript ist unter den Namen von Matthias
Günther veröffentlicht.)}
\newcommand{\publisher}{Der Helex-Matze Verlag $\sum\limits_{i=1}^{n}i$}
\newcommand{\pdfemail}{matthias.guenther@wikimatze.de}
\newcommand{\correctiontext}{Kommentare/Korrekturen an}
\newcommand{\homepagetext}{Homepage}
\newcommand{\homepage}{wikimatze.de}
\newcommand{\coverdisclaimer}{Copyright Skript-Covers}
\newcommand{\covercopyright}{\textsc{Ubisoft} (\url{ubi.com})}

%% fancy-header
\input{\home/style_header_oneside}

%% setting the infos for the pdf
\input{\home/info_hypersetup}

%% environments
\input{\home/environments_normal}
\input{\home/environments_mathe}

%% cover
\input{\home/style_cover}

\begin{document}
\input{\home/style_starting_document_without_cover}

\section{Kanban}
Hier steht was ...

\begin{itemize}
  \item Kan-ban entspricht Signalkarten: in Fertigung wird vorgelagerten Prozessschritt mitgeteilt,
    dass mehr Zwischenprodukte prod. werden sollen
  \item \textbf{Zusammenfassung} Kanban erlaubt eine kontextabhängige Optimierung Ihrer Prozesse bei
    minimalen Widerstand gegen diesen Prozess vorzunehmen
  \item ist ein Vorgehen um evolutionäre Änderungen herbeizuführen
  \item Kanban ist keine Softwareentwicklungsmethode und kein Projektmanagementansatz

    $\Rightarrow$  es setzt voraus, dass ein Prozess vorhanden ist und dieser Prozess soll
    inkrementell $\Delta$ werden
  \item \enquote{One size fits all} Entwicklungsmethoden funktionieren nicht:
    \begin{Beschreibungfett}[Projekte]
      \item [Projekte] Fähigkeiten, Erfahrungen, Leistungsvermögen
      \item [Projekte] Budget, Zeitpläne, Umfang, Risikoprofile
      \item [Orgs] Zielmarkt, Wertschöpfungsketten
    \end{Beschreibungfett}
    $\Rightarrow$ jedes Team zeichnet sich durch ein individuelles Bündchen von Möglichkeiten, Fähigkeiten und
    Erfahrungen aus!
  \item Vorschläge für $\Delta$ die nicht zum jeweiligen Kontext passen, werden von Menschen
    zurückgewiesen, die in diesem Projekt Kontext arbeiten und ihn verstehen

    $\Rightarrow$  wenn die vorgeschlagene Änderungen hinsichtlich der Arbeitspraktiken und
    Verhaltensweisen keine Vorteile bringen, dann lehnen die Menschen diese Veränderungen ab
  \item Pull $\Rightarrow$ Wip $\Rightarrow$ Schützen vor Überlastung
\end{itemize}


\subsection{Eigenschaften}
\begin{itemize}
  \item es gibt eine feste Anzahl an \textbf{Kanban-Karten},  wobei die Karten-Menge nach oben durch
    eine maximale Kapazität des Systems beschränkt ist
  \item jede Karte ist ein \textbf{Signalmechanismus} und neue Arbeitseinheit darf nur angefangen
    werden, wenn eine neue Karte verfügbar ist
  \item jede neue Aufgabe muss in \textbf{Warteschlange} warten, bis wieder eine Karte verfügbar ist
  \item \textbf{Pull-System}: Neue Aufgaben werden nur dann ins System gezogen (Pull), wenn das
    System die Kapazitäten dafür bereitstellen kann, die Aufgabe auch abzuarbeiten

    $\Rightarrow$  Aufgaben werden nicht nach Bedarf ins System gedrückt (\textbf{push})
  \item Pull-System kann nicht überlastet werden, wenn Kapazität durch die richtige Anzahl von
    Kanban-Karten adäquat festgelegt ist
  \item \textbf{WIP}: Menge an begonnener Arbeit
  \item \textbf{througpout}: Team wird entsprechend seinem Durchsatz mit Aufgaben befüllt, so dass
    es diese Aufgaben auch abschließen kann
\end{itemize}


\subsubsection{Wozu ein Kanban-System}
\begin{itemize}
  \item \textbf{sustainable pace}: nachhaltiges Arbeitstempo
  \item Schwachstellen in der Leistungsfähigkeit aufzeigen
  \item Teams sollen sich auf das Beheben von Schwachstellen fokussieren, um einen stetigen Fluss
    der Aufgaben zu erreichen
  \item durch Sichtbarkeit von Qualitäts- und Prozessproblemen wird klar, welche Auswirkungen
    Fehler, Engpässe, Variabilität und ökonomische Kosten auf diesen Fluss sowie Resultate der
    Arbeit haben
  \item WIP fördert höhere Qualität und besser Leistungsfähigkeiten
  \item Kombination aus verbesserten Fluss und bessere Qualität hilft bei Durchlaufzeit-Verbesserung
    und Vorhersagbarkeit sowie Termintreue (\textbf{due date performance})
  \item durch Einführung regelmäßiger Releaserhytmus und konsistenter Einhaltung davon, hilft
    Kanban, Vertrauen zum Kunden sowie entlang der Wertschöpfungskette zu anderen Abteilungen,
    Lieferanten und nachgelagerten Partnern im Prozess aufzubauen
  \item Kulturelle $\Delta$ von Orgs:
    \begin{itemize}
      \item Probleme werden aufgedeckt und Orgs konzentriert sich auf Lösung dieser und Beseitigung
        negativer Auswirkungen in der Zukunft
      \item Orgs verbessert sich selber immer weiter und Mitarbeiter agieren kooperativ und
        vertrauensvoll zusammen und besitzen einen hohen Grad an Entscheidungsfreiheit
    \end{itemize}
\end{itemize}


\subsubsection{Grundprinzipien und Kerneigenschaften}
\ulbf{Grundprinzipien}

\begin{enumerate}
  \item Beginne dort, wo du dich im Moment befindest
  \item Komme mit anderen überein, dass inkrementelle, evolutionäre $\Delta$ angestrebt werden
  \item Respektiere den bestehenden Prozess sowie die existierenden Rollen, Verantwortlichkeiten und
    Berufsbezeichnungen
\end{enumerate}


\ulbf{Kerneigenschaften}

\begin{enumerate}
  \item Visualisieren den Fluss der Arbeit
  \item Führe WIP ein
  \item Führe Messungen zum Fluss durch und kontrolliere Ihn
  \item Mache die Regeln für den Prozess explizit
  \item Verwende Modelle, um Chancen für Verbesserungen zu erkennen
\end{enumerate}


\subsubsection{Resultierende Verhaltensweisen}
\begin{enumerate}
  \item Für jede Wertschöpfungskette wird der Prozesse individuelle zugeschnitten
  \item Rhythmen werden voneinander entkoppelt (Entwicklung ohne Iterationen)
  \item Reihenfolge der Aufgaben richtet sich nach den Verzugskosten
  \item Ausgelieferte Wert wird durch Einsatz von Serviceklassen optimiert
  \item Durch Zuordnung von Kapazitäten wird Risikomanagement betrieben
  \item Es ist erlaubt mit dem Prozess zu experimentieren
  \item Quantitatives Management wird betrieben
  \item Kanban verbreitet sich wie ein Virus über die gesamte Orgs
  \item Kleine Teams verschmelzen miteinander, um flexible bei der Arbeit zu sein
\end{enumerate}


\section{Vorteile von Kanban}

\subsection{Erfolgsrezept}
Erfolgsrezept besteht aus folgenden 6 Schritten:

\begin{itemize}
  \item Fokussiere auf Qualität:
    \begin{itemize}
      \item TDD
      \item gemeinsames Analyse und Design
      \item Anwendung von Design-Patterns
      \item moderne Tools verwenden
      \item Pair-Programming
      \item Codereviews
    \end{itemize}
  \item Reduziere WIP und liefere häufig
    \begin{itemize}
      \item es besteht ein kausaler Zusammenhang zwischen der Menge an WIP und der durchschnittlichen
        Laufzeit $\Rightarrow$  linear  \textbf{Littles Law}
      \item Fehler steigen übermäßig an, sobald sich die Menge des WIPs erhöht
      \item wenn Ressourcen immer voll ausgelastet sind wird das Entstehen einer Kultur der Verbesserung
        behindert
      \item es ergibt wenig Sinn, sich der Priorisierung zuzuwenden, wenn es keine Vorhersagbarkeit
        bzgl. der Auslieferung gibt
      \item Rhythmus (\textit{cadence}) legt Taktik bestimmter Ereignisse fest
    \end{itemize}
  \item Sorge für ein Gleichgewicht zwischen Nachfrage und Durchsatz
    \begin{itemize}
      \item dann ist nur noch der Engpass voll ausgelastet
      \item Mitarbeiter haben freie Kapazitäten
      \item Freiräume: Arbeitsplatz aufräumen, Umfeld verbessern, neue Werkzeuge, Art der
        Zusammenarbeit mit anderen Kollegen
    \end{itemize}
  \item Priorisiere:

  \item Nimm Quellen von Variabilität ins Visier, um die Vorhersagbarkeit zu verbessern
    \begin{itemize}
      \item Variabilität führt zu mehr VIP und längere Durchlaufzeiten
    \end{itemize}
\end{itemize}


\section{Unsorted}
\begin{itemize}
  \item \textbf{Engpass}: muss solange erweitert werden, bis er die Gesamtleistung des Systems nicht
    länger einschränkt
  \item \textbf{Drum-Buffer-Rope}: Ansatz fürs Pull-System und Theory of Constraints
  \item mit Kanban führt man ein komplexes und adaptives System ein, dass dafür gedacht ist, Orgs in
    Richtung \itbf{Lean} zu entwickeln
\end{itemize}

\end{document}

