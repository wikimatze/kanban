%% config
\def\home{../../styles}

%% documentclass
\input{\home/documentclass_normal_oneside}

%% generell-styling
\input{\home/style_proggen}

%% meta-tags for pdf
\newcommand{\pdfauthor}{Matthias Günther}
\newcommand{\pdftitle}{Kanban}
\newcommand{\pdfsubject}{Aufzeichnungen zur Fortbildung}
\newcommand{\pdfkeywords}{ruby}
\newcommand{\motto}{Kanban ist eventuell besser als Scrum}
\newcommand{\tutor}{}
\newcommand{\disclaimer}{(Die Autoren übernehmen keine Garantie und Haftung
für die Korrektheit des Skriptes. Das Skript ist unter den Namen von Matthias
Günther veröffentlicht.)}
\newcommand{\publisher}{Der Helex-Matze Verlag $\sum\limits_{i=1}^{n}i$}
\newcommand{\pdfemail}{matthias.guenther@wikimatze.de}
\newcommand{\correctiontext}{Kommentare/Korrekturen an}
\newcommand{\homepagetext}{Homepage}
\newcommand{\homepage}{wikimatze.de}
\newcommand{\coverdisclaimer}{Copyright Skript-Covers}
\newcommand{\covercopyright}{\textsc{Ubisoft} (\url{ubi.com})}

%% fancy-header
\input{\home/style_header_oneside}

%% setting the infos for the pdf
\input{\home/info_hypersetup}

%% environments
\input{\home/environments_normal}
\input{\home/environments_mathe}

%% cover
\input{\home/style_cover}

\begin{document}
\input{\home/style_starting_document_without_cover}

\section{Kanban}
Hier steht was ...

\begin{itemize}
  \item Kan-ban entspricht Signalkarten: in Fertigung wird vorgelagerten Prozessschritt mitgeteilt,
    dass mehr Zwischenprodukte prod. werden sollen
  \item ist ein Vorgehen um evolutionäre Änderungen herbeizuführen
  \item \enquote{One size fits all} Entwicklungsmethoden funktionieren nicht:
    \begin{Beschreibungfett}[Projekte]
      \item [Projekte] Fähigkeiten, Erfahrungen, Leistungsvermögen
      \item [Projekte] Budget, Zeitpläne, Umfang, Risikoprofile
      \item [Orgs] Zielmarkt, Wertschöpfungsketten
    \end{Beschreibungfett}
    $\Rightarrow$ jedes Team zeichnet sich durch ein individuelles Bündchen von Möglichkeiten, Fähigkeiten und
    Erfahrungen aus!
  \item Vorschläge für $\Delta$ die nicht zum jeweiligen Kontext passen, werden von Menschen
    zurückgewiesen, die in diesem Projekt Kontext arbeiten und ihn verstehen

    $\Rightarrow$  wenn die vorgeschlagene Änderungen hinsichtlich der Arbeitspraktiken und
    Verhaltensweisen keine Vorteile bringen, dann lehnen die Menschen diese Veränderungen ab
  \item Pull $\Rightarrow$ Wip $\Rightarrow$ Schützen vor Überlastung
\end{itemize}


\section{Unsorted}
\begin{itemize}
  \item \textbf{Engpass}: muss solange erweitert werden, bis er die Gesamtleistung des Systems nicht
    länger einschränkt
  \item \textbf{Drum-Buffer-Rope}: Ansatz fürs Pull-System und Theory of Constraints
\end{itemize}

\end{document}

